% See https://de.overleaf.com/learn/latex/Glossaries

\newcommand{\newacronymalwaysshort}[3]{
\newglossaryentry{#1}{name={#2}, text={#2}, first={#2}, description={#3}, type=\acronymtype}
}

% Acronyms

\newacronym{ODM}{ODM}{Outlier detection method}
\newacronymalwaysshort{SQL}{SQL}{Structured Query Language, a language for accessing databases}
\newacronymalwaysshort{HTML}{HTML}{HyperText Markup Language, a language to describe the structure of websites}
\newacronymalwaysshort{CSS}{CSS}{Cascading Style Sheets, a language used to describe the appearance of a website}
\newacronymalwaysshort{CSV}{CSV}{Comma-separated values, a file format commonly used for storing datasets}
\newacronymalwaysshort{GUI}{GUI}{Grahpical User Interface}


% Glossary entries

\newglossaryentry{subspace-logic}
{
    name={Subspace logic},
    text={subspace logic},
    description={A logic that specifies in which \glspl*{subspace} a point of the dataset must be identified as an outlier in order to be contained in the \gls*{experiment} result}
}

\newglossaryentry{subspace}
{
    name={Subspace},
    text={subspace},
    description={A subspace of the complete dataset}
}

\newglossaryentry{datapoint}
{
    name={Data point},
    text={data point},
    description={A single entry of a dataset},
}

\newglossaryentry{index}
{
    name={Index of a \gls*{datapoint}},
    text={index},
    plural={indices},
    description={The number of the row in which the \gls*{datapoint} is located in the given dataset}
}

\newglossaryentry{experiment}
{
    name={Experiment},
    text={experiment},
    description={A configurable execution of outlier detection methods on \glspl*{subspace} of a dataset}
}

\newglossaryentry{hpp}
{
    name={Hyperparameter},
    text={hyperparameter},
    description={Parameters that an \gls*{ODM} requires}
}

\newglossaryentry{ground-truth-file}
{
    name={Ground-truth file},
    text={ground-truth file},
    description={\gls*{CSV} file containing which points of a dataset are actually outliers}
}

\newglossaryentry{subspace-outlier}
{
    name={Subspace outlier},
    text={subspace outlier},
    description={A datapoint that is an outlier in a given \gls*{subspace}}
}

% frameworks and libraries

\newglossaryentry{PyOD}
{
    name={PyOD},
    description={Python library providing methods for outlier detection},
}

\newglossaryentry{overleaf}
{
    name={Overleaf},
    description={Collaborative \LaTeX\ editor},
}

\newglossaryentry{texlive}
{
    name={TexLive},
    description={A distribution of the \TeX\ typesetting system used by \gls*{overleaf}},
}

\newglossaryentry{typescript}
{
    name={TypeScript},
    description={Programming language that can be used for web development}
}

\newglossaryentry{vue-js}
{
    name={Vue.js},
    description={Front-end framework that can be used for building websites}
}

\newglossaryentry{postgresql}
{
    name={PostgreSQL},
    description={A database system},
}

\newglossaryentry{python}
{
    name={Python},
    description={Programming language commonly used in the field of machine learning and outlier detection}
}

\newglossaryentry{pandas}
{
    name={pandas},
    description={\Gls*{python} library used for data analysis}
}

\newglossaryentry{flask}
{
    name={Flask},
    description={\Gls*{python} web framework}
}

\newglossaryentry{git}
{
    name={Git},
    description={A distributed version control system},
}

\newglossaryentry{gitlab}
{
    name={GitLab},
    description={Web application for version control based on \gls*{git}},
}
