\section{Routing/URL overview}

\subsection{Front-end}

Routing on the front-end side is implemented using the Vue router, which allows the definition of routes, which are key-value dictionaries containing (at the minimum) a path key and a component key. This way depending on the url of the browser Vue knows which component should be loaded. The passed component is rendered, when <RouterView /> is inserted into the template of a vue page/view.
Navigation guards can also be set up for the router. For example, a beforeEach guard can be registered, which as the name implies, is run before each redirection. This way, an unauthenticated user can be redirected to login page when trying to access the dashboard page via "/dashboard".  For this routes can be provided with meta fields such as "requiresAuth:true" to indicate that the route shall only be accessible, when the user is authenticated. 

\label{ssec:routing-front-end}
\subsubsection*{URL overview:}
\begin{itemize}
    \item /home $\Rightarrow$ HomeView loaded 
    \item /login $\Rightarrow$ LoginView loaded  
    \item /register $\Rightarrow$ RegisterView loaded  
\end{itemize}























































































































































































































































































































































\subsubsection*{Navigation guards:}
.beforeEach:
\begin{itemize}
    \item check whether "/login" or "/register" are attempted to be accessed by authenticated users and if that is the case redirect them to their dashboard
    \item redirect users to the login page if a route with the meta field "requiresAuth:true" is attempted to be accessed 
    \item in the case that the "/experiment-result/..." is attempted to be accessed first check whether an experiment with the provided experiment id exists.
\end{itemize}

\subsection{Back-end}
With routing on the back-end side, the API calls/endpoints are meant, which are handled using Flask. Since some API calls such as the "/dashboard" api call only make sense, when a user is logged in they are only handled when a valid access token is passed in the HTTP request's header. For this, a python wrapper function"@jwt-required()" is used. All routes prefix "/api".
\label{ssec:routing-back-end}
\subsubsection*{API endpoints:}
User API:
\begin{itemize}
    \item /user/login: Returns an access token if the username and password are correct.
    \item /user/register: If valid register data was passed a user account is created and an access token is returned.
    \item /user/logout: The user's access token is invalidated. Requires an access token.
\end{itemize}
Experiment API:
\begin{itemize}
    \item /experiment/validate-dataset: Checks whether the specified dataset is valid. Requires an access token.
    \item /experiment/validate-ground-truth: Checks whether the ground truth file is valid. Requires an access token.
    \item /experiment/get-result/<int:exp-id>: Returns the result of an experiment if an experiment with id <exp-id> exists and has finished executing. Requires an access token.
    \item /experiment/get-all: Returns a list of all the experiments the user has run. Requires an access token.
    \item /experiment/create: Creates and runs a new experiment. Requires an access token.
    \item /experiment/download-result/<int:exp-id>: Checks if the user has an experiment with the specified <exp-id>. Downloads the outliers with the applied subspace logic as a CSV file. Requires an access token.
\end{itemize}
ODM API:
\begin{itemize}
    \item /odm/get-all: Returns all of the availabe odms. Requires an access token.
    \item /odm/get-parameters/<int:odm-id>: Returns a list of all parameters the odm with id <odm-id> has including their type and whether they are optional. Requires an access token.
\end{itemize}