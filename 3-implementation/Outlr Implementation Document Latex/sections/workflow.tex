\section{Workflow}

\subsection*{Project management}
Our project is divided into tasks (see \ref{sec:timeline}).
Most tasks are the implementation of a package or sometimes individual classes. The goal is that all tasks need a similar amount of time to complete.

The dev branch is the main branch of the project and it is protected.
Each task is implemented on a separate feature branch that gets merged into the dev branch using a merge request.

Each task goes through the following phases:
\begin{enumerate}
    \item Not started: Task is not currently worked on
    \item In progress: The assigned team member is currently working on the task on a feature branch. Once the task is ready a merge request is created.
    \item Review: The task is ready, but needs to be reviewed by the assigned reviewer. The following requirements must be fulfilled before the merge request is accepted:
    \begin{itemize}
        \item The implemented feature should work as defined in the design phase
        \item Type annotations should be used in method signatures
        \item Documentation in the form of docstrings or TSDoc should be present on all public classes, attributes, and methods
        \item Basic unit tests should be implemented and all unit tests must pass
        \item Linter tests must pass
        \item There should be no merge conflicts
    \end{itemize}
    If a task does not fulfill these requirements it goes back to the previous phase.
    \item Done: The task is completed
\end{enumerate}
We use kanban boards in \gls{notion} to manage this workflow.

\subsection*{Continuous integration}
Our git repository is managed on \gls{gitlab}.
Continuous integration automates some parts of the review process.
Once a commit is pushed to a feature branch of the repository the app is built, the database is set up, and unit tests, as well as a lint test, is run automatically.
If any of these steps fail the merge request of the feature branch cannot be merged to the dev branch.

\begin{figure}[!ht]
    \centering
    \includesvg[width=\textwidth]{images/workflow-light-variant}
    \caption{Task phases}
    \label{fig:workflow}
\end{figure}
